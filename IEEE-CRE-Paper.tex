\documentclass[conference]{IEEEtran}
\IEEEoverridecommandlockouts
% The preceding line is only needed to identify funding in the first footnote. If that is unneeded, please comment it out.
%Template version as of 6/27/2024

\usepackage{cite}
\usepackage{amsmath,amssymb,amsfonts}
\usepackage{algorithmic}
\usepackage{graphicx}
\usepackage{textcomp}
\usepackage{xcolor}
\def\BibTeX{{\rm B\kern-.05em{\sc i\kern-.025em b}\kern-.08em
    T\kern-.1667em\lower.7ex\hbox{E}\kern-.125emX}}
\begin{document}

\title{Classification of PGN\\
%{\footnotesize \textsuperscript{*}}
%\thanks{Identify applicable funding agency here. If none, delete this.}
}

\author{\IEEEauthorblockN{Logan Miller}
\IEEEauthorblockA{\textit{Departments of Math and Computer Science}\\
\textit{Lawrence Technological University}\\
Southfield, MI, USA \\
lmiller2@ltu.edu}
\and
\IEEEauthorblockN{Rory Wilson}
\IEEEauthorblockA{\textit{Department of Math and Computer Science} \\
\textit{Lawrence Technological University}\\
Southfield, MI, USA \\
rwilson2@ltu.edu}


}

\maketitle

\begin{abstract}
Make this a summarization of your paper.Make this a summarization of your paper.Make this a summarization of your paper.Make this a summarization of your paper.Make this a summarization of your paper.Make this a summarization of your 
\end{abstract}

\begin{IEEEkeywords}
PGN, Clustering, Chess.
\end{IEEEkeywords}

\section{Introduction}
Chess is one of the oldest and most universally played strategy games, and with the advent of online platforms like chess.com, millions of games are now available for analysis. These platforms generate a wealth of data, which presents an opportunity to gain insights into player behavior, skill levels, and strategies. However, analyzing such large datasets to extract meaningful patterns remains a challenge. While traditional methods often focus on broad metrics like Elo ratings, they fail to account for the nuanced differences in play styles and strategies that define various skill levels. This paper aims to address this gap by employing machine learning techniques, specifically clustering, to categorize chess games based on player skill levels and game play characteristics, offering a more granular understanding of player performance.

The importance of this problem lies in its potential to enhance player development, improve training tools, and refine chess engines. Automatically classifying games by skill level can provide valuable feedback to players, enabling personalized improvement suggestions. Additionally, it offers coaches and analysts a more detailed view of game play, which can inform training strategies. The key challenges in this task include handling the complex, high-dimensional nature of chess data and identifying relevant features that effectively differentiate skill levels. The main contribution of this work is a novel clustering approach that groups games based on Elo ratings, move length, and game result, providing deeper insights into the relationship between game play and player expertise.

\section{Related Work}
\begin{itemize}
\item Previous approaches to the problem
\item Relevant methods and techniques
\item Limitations of existing work
\item How your work differs/improves upon prior work
\end{itemize}

\subsection{Make it your own}\label{AA}
Define abbreviations and acronyms the first time they are used in the text, 
even after they have been defined in the abstract. Abbreviations such as 
IEEE, SI, MKS, CGS, ac, dc, and rms do not have to be defined. Do not use 
abbreviations in the title or heads unless they are unavoidable.

\subsection{Make it your own}
\begin{itemize}
\item Use either SI (MKS) or CGS as primary units.
\item Avoid combining SI and CGS units,
\item Do not mix complete spellings 
\item Use a zero before decimal points:
\end{itemize}

\subsection{Make it your own}
Number equations consecutively. To make your 
equations more compact, you may use the solidus (~/~), the exp function, or 
appropriate exponents. Italicize Roman symbols for quantities and variables, 
but not Greek symbols. Use a long dash rather than a hyphen for a minus 
sign. Punctuate equations with commas or periods when they are part of a 
sentence, as in:
\begin{equation}
a+b=\gamma\label{eq}
\end{equation}

Be sure that the 
symbols in your equation have been defined before or immediately following 
the equation. Use ``\eqref{eq}'', not ``Eq.~\eqref{eq}'' or ``equation \eqref{eq}'', except at 
the beginning of a sentence: ``Equation \eqref{eq} is . . .''

\section{Methodology}
\subsection{Data Collection and Preprocessing}
To gather the data about the different chess games there is a website https://database.lichess.org/ that is filled with different games with different level of players. the data includes the move sequences, player ratings, game outcomes, and other metadata, which provides the foundation for clustering analysis. The data is cleaned by removing games with non-stnadard variants, incomplete results, or issing Elo ratings.

The preprocessing steps include splitting each game's metadata and move sequences, then filtering out games that do not meet the following criteria:
\begin{itemize}
    \item The game must be a standard chess game (no variants).
    \item The game must have a normal termination and result.
    \item The game must have valid Elo ratings for both players.
\end{itemize}

The cleaned data consits of three key features for each game: the White player's Elo rating, the number of moves made during the game, and the game results (encoded as 0, 0.5, or 1 for black win, draw, or white win, respectively)

\subsection{Feature Extraction}

The feature extraction process focuses on simplifying and organizing the data for use in clustering algorithms. The extraced features include:
\begin{itemize}
    \item Player Elo ratings: The rating of the white and black players
    \item Move length: The total number of moves made during the game.
    \item Game result: The outcome of the game, represented numerically for clustering purposes.
\end{itemize}
The moves are split into individual actions, and each game's result is encoded to represent the player's performance, either as a win, loss, to draw.

\subsection{Clustering Approach}



\section{Experimental Setup}
\subsection{Dataset}

\subsection{Evaluation Metrics}

\subsection{Baseline}

\subsection{Implementation Details}

\subsection{Hyper parameters}

\section{Results and Discussion}

\section{Conclusion}

\section{References}

\subsection{Figures and Tables}\label{FAT}
\paragraph{Positioning Figures and Tables} Place figures and tables at the top and 
bottom of columns. Avoid placing them in the middle of columns. Large 
figures and tables may span across both columns. Figure captions should be 
below the figures; table heads should appear above the tables. Insert 
figures and tables after they are cited in the text. Use the abbreviation 
``Fig.~\ref{fig}'', even at the beginning of a sentence.

\begin{table}[htbp]
\caption{Table Type Styles}
\begin{center}
\begin{tabular}{|c|c|c|c|}
\hline
\textbf{Table}&\multicolumn{3}{|c|}{\textbf{Table Column Head}} \\
\cline{2-4} 
\textbf{Head} & \textbf{\textit{Table column subhead}}& \textbf{\textit{Subhead}}& \textbf{\textit{Subhead}} \\
\hline
copy& More table copy$^{\mathrm{a}}$& &  \\
\hline
\multicolumn{4}{l}{$^{\mathrm{a}}$Sample of a Table footnote.}
\end{tabular}
\label{tab1}
\end{center}
\end{table}

\begin{figure}[htbp]
\centerline{\includegraphics{overleaf logo.png}}
\caption{Example of a figure caption.}
\label{fig}
\end{figure}

\includegraphics[scale = 1, ]
{overleaf logo.png}

\section{results and discussion}

\section{Conclusion}

Figure Labels: Use 8 point Times New Roman for Figure labels. Use words 
rather than symbols or abbreviations when writing Figure axis labels to 
avoid confusing the reader. As an example, write the quantity 
``Magnetization'', or ``Magnetization, M'', not just ``M''. If including 
units in the label, present them within parentheses. Do not label axes only 
with units. In the example, write ``Magnetization (A/m)'' or ``Magnetization 
\{A[m(1)]\}'', not just ``A/m''. Do not label axes with a ratio of 
quantities and units. For example, write ``Temperature (K)'', not 
``Temperature/K''.

\begin{thebibliography}{00}
\bibitem{b1} F. Wijayanto, "Clustering Analysis of Chess Protable Game Notation Text," Jurnal Sians, Nalar, Dan Aplikasi Teknologi Informasi, 3(3), pp. 137--142, 2024.
\bibitem{b2} J. Clerk Maxwell, A Treatise on Electricity and Magnetism, 3rd ed., vol. 2. Oxford: Clarendon, 1892, pp.68--73.
\bibitem{b3} I. S. Jacobs and C. P. Bean, ``Fine particles, thin films and exchange anisotropy,'' in Magnetism, vol. III, G. T. Rado and H. Suhl, Eds. New York: Academic, 1963, pp. 271--350.
\bibitem{b4} K. Elissa, ``Title of paper if known,'' unpublished.
\bibitem{b5} R. Nicole, ``Title of paper with only first word capitalized,'' J. Name Stand. Abbrev., in press.
\bibitem{b6} Y. Yorozu, M. Hirano, K. Oka, and Y. Tagawa, ``Electron spectroscopy studies on magneto-optical media and plastic substrate interface,'' IEEE Transl. J. Magn. Japan, vol. 2, pp. 740--741, August 1987 [Digests 9th Annual Conf. Magnetics Japan, p. 301, 1982].
\bibitem{b7} M. Young, The Technical Writer's Handbook. Mill Valley, CA: University Science, 1989.
\bibitem{b8} D. P. Kingma and M. Welling, ``Auto-encoding variational Bayes,'' 2013, arXiv:1312.6114. [Online]. Available: https://arxiv.org/abs/1312.6114
\bibitem{b9} S. Liu, ``Wi-Fi Energy Detection Testbed (12MTC),'' 2023, gitHub repository. [Online]. Available: https://github.com/liustone99/Wi-Fi-Energy-Detection-Testbed-12MTC
\bibitem{b10} ``Treatment episode data set: discharges (TEDS-D): concatenated, 2006 to 2009.'' U.S. Department of Health and Human Services, Substance Abuse and Mental Health Services Administration, Office of Applied Studies, August, 2013, DOI:10.3886/ICPSR30122.v2
\end{thebibliography}

\end{document}
